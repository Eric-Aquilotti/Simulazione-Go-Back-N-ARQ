\documentclass{article}
\usepackage[utf8]{inputenc}
\usepackage{listings}
\usepackage{color}
\usepackage{hyperref}

\title{
    Elaborato programmazione di reti:
    
    Simulazione Go-Back-N ARQ
}
\author{Eric Aquilotti}
\date{\today}

\begin{document}

\maketitle

\section{Introduzione}
Il seguente progetto implementa una simulazione del protocollo
\textbf{Go-Back-N ARQ} per l'invio affidabile di pacchetti tramite UDP. UDP non
garantisce affidabilità, quindi si rende necessario un meccanismo di controllo
degli errori a livello applicativo.

\section{Struttura del \texttt{server.py}}

Il server mantiene un contatore globale \texttt{expected} che rappresenta il
numero del prossimo pacchetto atteso. Alla ricezione di un pacchetto:
\begin{itemize}
    \item Il pacchetto viene decodificato e confrontato con il valore atteso.
    \item Se corrisponde, viene inviato un ACK al client e \texttt{expected}
        viene incrementato.
    \item Il server simula inoltre:
    \begin{itemize}
        \item Ritardi casuali (5\% dei casi), tramite \texttt{time.sleep(2)}.
        \item Errori di trasmissione (5\% dei casi), in cui il pacchetto viene
            ignorato.
    \end{itemize}
\end{itemize}

\section{Struttura del \texttt{client.py}}

Il client implementa il protocollo Go-Back-N:
\begin{itemize}
    \item Invia i pacchetti in base a una \textbf{finestra scorrevole} di
        dimensione \texttt{window\_size}.
    \item Tiene traccia del \texttt{base} (primo pacchetto non confermato) e del
        \texttt{current} (prossimo pacchetto da inviare).
    \item Se scade il timeout prima della ricezione di un ACK, tutti i pacchetti
        non confermati nella finestra vengono ritrasmessi.
    \item L'uso di \texttt{select.select} permette di gestire la ricezione
        asincrona degli ACK con timeout.
    \item Un ACK valido è accettato solo se corrisponde a \texttt{base}. In caso
        contrario, è ignorato.
    \item Statistica: A fine trasmissione, vengono stampati i pacchetti
        persi e il numero di ritrasmissioni effettuate.
\end{itemize}

\end{document}
